\documentclass{article}

% preamble, set properties here
\usepackage{amsmath}  % for custom math
\usepackage{graphicx} % for figures
\usepackage{parskip}  % better looking paragraphs


\title{Over-Explicit Lie Algebra Guide}
\date{\today}
\author{Jeffrey Millard}

\begin{document}

%\pagenumbering{gobble}
\maketitle
%\newpage
%\pagenumbering{arabic}

This is my attempt at understanding Lie algebra.

\textsc{The aim of this paper is to eventually understand the "special linear group" for the sake mastering homography estimation.}


\section{Unit Circle}
  This section covers Jacob White's first few presentation slides explaining 2D Lie algebra example. Consider a point on the unit circle. The goal is to interpolate smoothly between two different points on the unit circle without using trigonometry.

  \begin{equation}
    v_i = a_i + jb_i
  \end{equation}

  Since this is on the unit circle, it can be parameterized by a single angle $\theta$. To demonstrate the desired smooth transition between angles, we write

  \begin{equation}
    \theta(t) = \theta_1 + (\theta_2-\theta_1)t
  \end{equation}

  For the answer, we'd like to write $v(t)$ as a function of $\theta$.

  \begin{equation}
    v = f(\theta(t))
  \end{equation}

  Without diving into the theory, we can use Euler's formula which allows us to move along the $\theta$ space, and express the result in the $v$ space.

  \begin{equation}
    v(t) = e^{j\theta(t)} = \cos(\theta(t)) + j\sin(\theta(t)) = a(t) + jb(t)
  \end{equation}

  We skipped to the answer using a known formula without touching the theory, but the principle is demonstrated. We can move smoothly along $\theta$, but represent the result in the original format (by inspection we know the $a$'s and $b$'s don't change linearly with time). In this case, transitioning back and forth between $\theta$ space and $v$ space is a matter of taking a natural log. Similar operations will be seen later.

\section{2D Rotations}
  In the example above, we expressed a point $v$ in terms of $\theta$ which was linearly changing with respect to time. Clearly the same $\theta$ can parameterize a 2D rotation matrix:

  \begin{equation}
    R = \left[\begin{matrix} \cos(\theta) & -\sin(\theta) \\
                             \sin(\theta) & \cos(\theta) \end{matrix}\right]
  \end{equation}

  Can we come up with a transition formula that takes us from the $\theta$ space to a rotation matrix without using trigonometry? One might argue that the expression above is a perfectly valid transition formula despite the fact it uses trigonometry. Even if we permitted the use of basic trigonometric functions in this toy problem, briefly consider the case of 3D rotations. Let us define the smooth rotation over time as a single rotation angle about the axis of rotation (because it's intuitive to think about smooth changes). Clearly, a 3D version of the above expression would be very complicated and/or inadequate.

  We can use the following matrix representation of complex numbers. For this to be valid, it is important that $J^2 = -I$.

  \begin{equation}
    I = \left[\begin{matrix} 1 & 0 \\
                             0 & 1 \end{matrix}\right]
  \end{equation}
  \begin{equation}
    J = \left[\begin{matrix} 0 & -1 \\
                             1 & 0  \end{matrix}\right]
  \end{equation}

  Euler's formula with this substitution yields

  \begin{equation}
    e^{J\theta} = I\cos(\theta) + J\sin(\theta) = \\
    \left[\begin{matrix} \cos(\theta) & -\sin(\theta) \\
                         \sin(\theta) & \cos(\theta) \end{matrix}\right] = \\
    R
  \end{equation}

  This provides a simple expression for the rotation matrix that doesn't require evaluation of trigonometric functions.

  \begin{equation}
    R = e^{J\theta}
  \end{equation}
  \begin{equation}
    J\theta = \ln(R)
  \end{equation}

  Again, we skipped to the answer using a known formula (and a clever matrix substitution). This provides another example of the principle. We can move smoothly along $\theta$, but represent the result in a rotation matrix (which has very little rotational intuition by looking at it). In this case, transitioning back and forth is a matter of taking a matrix natural log.

\section{Lie Group vs. Lie Algebra}
  We've covered some basic concepts with some simple examples. Certain elements (such as 2D rotation matrices) form a group. A group requires:
  \begin{itemize}
    \item Closure. If $R_1$ and $R_2$ are members of the group, $R_1R_2$ is also a member.
    \item Associativity. $(R_1R_2)R_3 = R_1(R_2R_3)$
    \item Identity. $R_1I = R_1$
    \item Inverse. $RR^{-1}=R^{-1}R=I$
  \end{itemize}
  A Lie group is a group that is also a smooth differentiable manifold. When the Lie group is replaced with a local linearized version, the result is its Lie algebra. For this reason, the term "tangent space" is sometimes used in relation to the Lie algebra.

  Some clarification about manifolds: In the example above, it can be tempting to think of $\theta$ as the manifold because we can move smoothly along it. While this is true, the Lie group of rotation matrices is the same smooth manifold. The key for understanding here is that $\theta$ is a minimal realization and also happens to forms a vector space, which makes operations easier. Hence, the smooth interpolation was performed in $\theta$ space.

  Therefore, given an element of a Lie group (rotation matrix, homography, etc), we can transform it to a minimal realization (which lives in a linear vector space). Operations can then be performed smoothly. So far, this is all we've done with the 2D rotation matrix example. Strictly speaking, operations on $\theta$ are operations on the Lie group. The tangent space for a given position on the manifold constitutes a Lie algebra and operations on it result in an approximation (which needs to be projected back onto the manifold afterwards).

\section{Finding the Lie Algebra}
  I'm still figuring this out. Jacob says "the Lie algebra is the tangent space of the rotation matrices at the identity element" Why at the identity?

  Why this is, needs to be explained better to me, but let's continue with our 2D example. Linearize the rotation matrix.

  \begin{equation} \label{eq:lie_algebra}
    R = \left[\begin{matrix} \cos(\theta) & -\sin(\theta) \\
                             \sin(\theta) & \cos(\theta) \end{matrix}\right] \\
      \approx I + \left[\begin{matrix} 0      & -\theta \\
                                       \theta & 0       \end{matrix}\right]
  \end{equation}

  Thus the tangent space, or Lie Algebra, is
  \begin{equation}
    \omega_x = \left[\begin{matrix} 0      & -\theta \\
                                    \theta & 0       \end{matrix}\right]
  \end{equation}

  Something about the generators, which I need to stare at longer.

  Linear vector space is
  \begin{equation}
    \omega = \left[\begin{matrix} \theta \end{matrix}\right]
  \end{equation}

  Let's say we want to make a 2D rotation by an arbitrary value of $\theta$. In the 2D toy problem, we can do this easily enough by operating on the Lie group. For the sake of learning, let's insist that we have to make the operation using the Lie algebra. Substituting the $\theta$ value into (\ref{eq:lie_algebra}) we get

  \begin{equation}
    R \approx I + X
  \end{equation}

  where $X$ is the Lie algebra with the substituted value of $\theta$, the desired amount of rotation. Clearly, the result isn't a true rotation matrix and large values of $\theta$ will produce large inaccuracies. However, if the rotation is applied in smaller pieces, the error can be decreased.

  \begin{equation}
    R \approx \left( I + \frac{X}{3} \right) \left( I + \frac{X}{3} \right) \left( I + \frac{X}{3} \right) = \left( I + \frac{X}{3} \right)^3
  \end{equation}

  Expanding to the general case, we get the definition of the matrix exponential.

  \begin{equation}
    R = \lim_{N\to\infty} \left( I + \frac{X}{N} \right)^N = \exp(X)
  \end{equation}

  This is how one can use the Lie algebra to perform operations on a Lie group (manifold). The matrix exponential can be computationally expensive; see Rodrigues' formula for simplifying the operation in the case of of skew symmetric Lie algebras.

  (maybe add notes about Rodriquez Formula)
  (add Python or Matlab example for 2d)
  (add notes for 3d)


  % % example figure
  % \begin{figure}
  %   \includegraphics[width=\linewidth]{figures/fig1.png}
  %   \caption{Description here.}
  %   \label{fig:tag1}
  % \end{figure}
  %
  % Figure \ref{fig:tag1} shows that...

\end{document}